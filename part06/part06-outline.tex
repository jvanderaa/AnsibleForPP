\documentclass[aspectratio=169]{beamer}
\usepackage{fontawesome}
\usepackage{hyperref}
\usepackage{url}
\usetheme{Madrid}
\usecolortheme{sidebartab}
\usefonttheme{professionalfonts}
\urlstyle{same}

\title{Ansible for Network Automation}
\subtitle{Templating with Ansible}
\date{}
\author{Josh VanDeraa}

\begin{document}
\begin{frame}
  \maketitle
  \footnotesize
  \faTwitter vanderaaj \hfill \faGithub jvanderaa \hfill \faSlack jvanderaa
\end{frame}

\begin{frame}
    \frametitle{Session Overview}
    At the end of this session you will:
    \begin{itemize}
      \item <2-> Have worked with templates to build various configuration stanzas
      \item <3-> Have seen various methods for working with Jinja2 templates in Ansible
      \begin{itemize}
          \item <3-> Template from Ansible task to output
          \item <4-> Template from Ansible task to a file
          \item <5-> Template from a file to a file
          \item <6-> Template from a file to an IOS device
      \end{itemize}
    \end{itemize}
  \end{frame}

  \begin{frame}[t]
    \frametitle{Jinja2 - Templating Language}
      \begin{center}
        \includegraphics[width=3cm]{assets/jinja-logo.png}
      \end{center}
      Ansible leverages Jinja2 templates for templating.
      \begin{itemize}
          \item Modelled after Django's (Python Web Framework) templating language
          \item Can be used in generic Python scripting as well
          \item Templating language for other Python based frameworks as well
      \end{itemize}
      \tiny
      \vfill
      image source (2019-11-10): \url{https://jinja.palletsprojects.com/en/2.10.x/}
  \end{frame}

  \begin{frame}
    \frametitle{DEMO!}
    \begin{columns}
    \begin{column}{0.3\textwidth}
      \Huge
      \begin{center}
        \faDesktop 
        \hspace{.5cm}
        \faRocket     
      \end{center}
    \end{column}
    \begin{column}{0.7\textwidth}
      \huge 
        Let's take a look!
        \begin{itemize}
          \item Jinja2 Templates
        \end{itemize}
    \end{column}
    \end{columns}
  \end{frame}

  \begin{frame}
    \frametitle{Summary}
      To review what we accomplished today:
      \begin{itemize}
        \item <2-> Have worked with templates to build various configuration stanzas
        \item <3-> Have seen various methods for working with Jinja2 templates in Ansible
        \begin{itemize}
            \item <3-> Template from Ansible task to output
            \item <4-> Template from Ansible task to a file
            \item <5-> Template from a file to a file
            \item <6-> Template from a file to an IOS device
        \end{itemize}
      \end{itemize}
  \end{frame}

  \begin{frame}
    \frametitle{Contact}
    \huge
    \begin{center}
      \url{https://packetpushers.slack.com}
    \end{center}
    \begin{center}
      \normalsize
      \faSlack \hspace{.1cm}jvanderaa  
    \end{center}
  \end{frame}

\end{document}